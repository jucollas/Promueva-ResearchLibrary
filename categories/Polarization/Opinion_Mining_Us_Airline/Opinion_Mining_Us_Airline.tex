\documentclass{article}
\usepackage{hyperref}
\usepackage{geometry}

\geometry{a4paper, margin=1in}

\title{Opinion Mining on US Airline Twitter Data Using Machine Learning Techniques}
\author{}
\date{}

\begin{document}

\maketitle

\section*{Paper Information}
\begin{itemize}
    \item \textbf{Paper link}: \url{https://doi.org/10.1109/ICENCO49778.2020.9357390}
    \item \textbf{Reviewer}: Juan José Viafara Carabali
    \item \textbf{Categories}: Polarization, Data
\end{itemize}

\section*{Addressed Problem}
How to perform sentiment analysis on social media using machine learning techniques?

\section*{Main Contribution}
A method that combines machine learning-based sentiment analysis with lexicon-based techniques.

\section*{Methodology Comments}
The approach uses Bag of Words and SVM to classify tweets related to US airlines as positive, negative, or neutral.

\section*{General Description}
This paper presents a sentiment analysis method applied to a dataset of US airline passengers’ tweets. The method used Bag of Words to represent preprocessed tweets as feature vectors and fed them into machine learning models. Support Vector Machine (SVM) yielded the best results, with an average accuracy of 85.59%.

\section*{Significance for the Project}
The approach outlined in this paper can contribute to sentiment analysis in social media, helping understand opinions related to public services or brands, which may align with Promueva’s goals in analyzing user sentiments and trends.

\end{document}
