\documentclass{article}
\usepackage{hyperref}
\usepackage{geometry}

\geometry{a4paper, margin=1in}

\title{Medición de la Influencia de Usuarios en Redes Sociales: Propuesta SocialEngagement}
\author{}
\date{}

\begin{document}

\maketitle

\section*{Paper Information}
\begin{itemize}
    \item \textbf{Paper link}: \url{http://dx.doi.org/10.1016/j.ipm.2016.04.003}
    \item \textbf{Reviewer}: Juan José Viafara Carabali
    \item \textbf{Categories}: Polarization, Data
\end{itemize}

\section*{Addressed Problem}
How to measure the influence and impact of users on social networks such as Twitter, Facebook, and Instagram?

\section*{Main Contribution}
Presentation of various indicators used to measure influence (indegree influence, retweet influence, mention influence, etc.) and introduction to the SocialEngagement platform.

\section*{Methodology Comments}
The paper provides an overview of metrics used on social media but does not explain how SocialEngagement calculates these indicators.

\section*{General Description}
This paper reviews multiple metrics used to measure influence on platforms like Twitter, Facebook, Instagram, and ResearchGate. It highlights tools such as Klout and introduces SocialEngagement, a platform that measures the importance of nodes and their influence on the rest of the network, though the specific calculations are not detailed.

\section*{Significance for the Project}
This research is valuable for Promueva as it explores metrics and platforms for measuring social media influence, offering insights into how user impact and engagement can be quantified to study polarization effects in social networks.

\end{document}
