\documentclass{article}
\usepackage{hyperref}
\usepackage{geometry}

\geometry{a4paper, margin=1in}

\title{The Influence of Joe Wicks on Physical Activity During the COVID-19 Pandemic: Thematic, Location, and Social Network Analysis of X Data}
\author{}
\date{}

\begin{document}

\maketitle

\section*{Paper Information}
\begin{itemize}
    \item \textbf{Paper link}: \url{http://dx.doi.org/10.2196/49921}
    \item \textbf{Reviewer}: Juan José Viafara Carabali
    \item \textbf{Categories}: Polarization, Opinion Evolution, Data
\end{itemize}

\section*{Addressed Problem}
How to measure the influence of public figures on social media during the COVID-19 pandemic?

\section*{Main Contribution}
A thematic and network analysis to measure Joe Wicks’ influence on physical activity during the pandemic, using betweenness centrality.

\section*{Methodology Comments}
The study uses betweenness centrality to analyze the flow of information and user influence on the network.

\section*{General Description}
The study collected over 290,000 posts from X related to Joe Wicks during the pandemic. Influence was estimated using betweenness centrality, which highlights the importance of nodes that serve as bridges connecting different users within the network.

\section*{Significance for the Project}
This study provides valuable insights into how influential individuals can impact public behavior during a crisis, which is relevant to Promueva's goal of understanding influence in social networks and examining how opinion shifts manifest through interactions during critical events.

\end{document}
