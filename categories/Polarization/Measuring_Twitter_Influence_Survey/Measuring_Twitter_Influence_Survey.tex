\documentclass{article}
\usepackage{hyperref}
\usepackage{geometry}

\geometry{a4paper, margin=1in}

\title{Measuring User Influence on Twitter: A Survey}
\author{}
\date{}

\begin{document}

\maketitle

\section*{Paper Information}
\begin{itemize}
    \item \textbf{Paper link}: \url{http://dx.doi.org/10.1016/j.ipm.2016.04.003}
    \item \textbf{Reviewer}: Juan José Viafara Carabali
    \item \textbf{Categories}: Polarization, Data
\end{itemize}

\section*{Addressed Problem}
How to measure user influence on Twitter?

\section*{Main Contribution}
A review of metrics for measuring activity, popularity, and influence on Twitter.

\section*{Methodology Comments}
Three categories of metrics are highlighted: activity (e.g., TweetRank), popularity (e.g., FollowerRank), and influence (e.g., closeness centrality). However, many of these metrics are not accessible due to PROMUEVA's limited access to X API constraints.

\section*{General Description}
This paper surveys influence metrics on Twitter up to 2015. It discusses traditional measures based on network position as well as new metrics that use Twitter-specific data. Many of these metrics are not available for this research due to the API's restricted access to certain user data.

\section*{Significance for the Project}
This paper provides a valuable overview of different metrics for measuring influence on Twitter. These insights can help analyze social dynamics in Promueva's project, even though the API limitations may restrict access to certain metrics.

\end{document}
