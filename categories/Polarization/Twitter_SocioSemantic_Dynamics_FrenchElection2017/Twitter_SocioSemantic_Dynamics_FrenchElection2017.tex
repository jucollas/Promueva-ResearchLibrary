\documentclass{article}
\usepackage{hyperref}
\usepackage{geometry}

\geometry{a4paper, margin=1in}

\title{Reconstruction of the Socio-Semantic Dynamics of Political Activist Twitter Networks: Method and Application to the 2017 French Presidential Election}
\author{}
\date{}

\begin{document}

\maketitle

\section*{Paper Information}
\begin{itemize}
    \item \textbf{Paper link}: \url{https://doi.org/10.1371/journal.pone.0201879}
    \item \textbf{Reviewer}: Juan José Viafara Carabali
    \item \textbf{Categories}: Polarization, Data
\end{itemize}

\section*{Addressed Problem}
How to detect political groups in social networks?

\section*{Main Contribution}
A method based on user interaction and independent of shared content.

\section*{Methodology Comments}
This method introduces a valuable strategy for estimating social influence.

\section*{General Description}
In this study, a network is constructed with undirected weighted edges representing the number of retweets between two nodes representing Twitter accounts. To detect political groups (non-overlapping communities), the Louvain algorithm is applied. This algorithm is based on edge weights to assemble communities, creating subgroups of accounts that share more unmodified information (retweets with each other) than with the rest of the network.

\section*{Significance for the Project}
This study is significant for understanding the dynamics of political polarization within social networks, which aligns with the objectives of Promueva. The methodology can help analyze user interaction patterns independent of content, providing insights into political group formation and influence.

\end{document}
