\documentclass{article}
\usepackage{hyperref}
\usepackage{geometry}

\geometry{a4paper, margin=1in}

\title{Unveiling Oligarchy in Influence Networks From Partial Information}
\author{}
\date{}

\begin{document}

\maketitle

\section*{Paper Information}
\begin{itemize}
    \item \textbf{Paper link}: \url{https://doi.org/10.1109/TCNS.2022.3225299}
    \item \textbf{Reviewer}: Juan José Viafara Carabali
    \item \textbf{Categories}: Polarization, Data
\end{itemize}

\section*{Addressed Problem}
How to study the formation of oligarchy in networks of influence?

\section*{Main Contribution}
A mathematical model to study the concentration of influence in social networks.

\section*{Methodology Comments}
Although the estimation of influence in real data is not explained, an algorithm for detecting oligarchy in an influence network is presented.

\section*{General Description}
This paper proposes a mathematical model to study oligarchy in networks of influence, where social power is highly concentrated among a few individuals. Additionally, it presents an algorithm to detect oligarchy within networks based on influence data.

\section*{Significance for the Project}
This study is relevant to Promueva's objectives as it provides a model for analyzing the concentration of influence in social networks, which can reveal insights into power dynamics and polarization trends within the Valle del Cauca's social landscape.

\end{document}
