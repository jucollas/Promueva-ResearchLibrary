\documentclass{article}
\usepackage{hyperref}
\usepackage{geometry}

\geometry{a4paper, margin=1in}

\title{Contextual Target-Specific Stance Detection on Twitter: Dataset and Method}
\author{}
\date{}

\begin{document}

\maketitle

\section*{Paper Information}
\begin{itemize}
    \item \textbf{Paper link}: \url{https://doi.org/10.1109/ICDM58522.2023.00045}
    \item \textbf{Reviewer}: Juan José Viafara Carabali
    \item \textbf{Categories}: Polarization, Data
\end{itemize}

\section*{Addressed Problem}
How to detect users' stances toward COVID-19 vaccination on Twitter?

\section*{Main Contribution}
A contextual model that leverages entire Twitter conversations, rather than individual tweets, to infer stances.

\section*{Methodology Comments}
The study uses Contextual Multi-Head Attention, which processes both the content of each tweet and the relationships within conversations to improve stance detection.

\section*{General Description}
This paper presents a dataset and a method to identify positions (favor, against, or neither) towards Covid vaccination of a twitter post using as context the conversation in which that tweet originated, which increased the F1 evaluation metric by 25 percentage points.

\section*{Significance for the Project}
The model and dataset proposed could enhance stance detection in polarized discussions, such as COVID-19 vaccination, providing useful insights for analyzing opinions and identifying polarization in social media.

\end{document}
