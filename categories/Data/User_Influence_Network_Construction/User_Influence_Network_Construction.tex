\documentclass{article}
\usepackage{hyperref}
\usepackage{geometry}

\geometry{a4paper, margin=1in}

\title{A User Influence Network Construction Approach Based on Web Mining and Social Network Analysis}
\author{}
\date{}

\begin{document}

\maketitle

\section*{Paper Information}
\begin{itemize}
    \item \textbf{Paper link}: \url{https://doi.org/10.1109/IEEM58616.2023.10406593}
    \item \textbf{Reviewer}: Juan José Viafara Carabali
    \item \textbf{Categories}: Data, Polarization
\end{itemize}

\section*{Addressed Problem}
How to construct an influence network based on user-generated content in a Chinese social network?

\section*{Main Contribution}
A strategy to infer interpersonal influence in digital social networks.

\section*{Methodology Comments}
Two valuable indicators, weight and influence, are deduced from the data, although for this research “weight” is more relevant as a measure of influence.

\section*{General Description}
This study proposes a method to build a network from user-generated content in RED, a Chinese social network where users post reviews of their purchases. A framework divided into three modules is presented:
\begin{enumerate}
    \item \textbf{Data Processing}: Data was obtained and cleaned.
    \item \textbf{Analysis}: Two important indicators, weight and influence, were inferred from the data.
    \item \textbf{Visualization}: Built from the dataset.
\end{enumerate}

Influential users were considered topic initiators and were defined by a significant number of followers, numerous likes, or frequent posts. The weight was used to measure the relationship between users by considering the similarity between texts and the number of interactions of each user. This weight quantified the strength of connections between editors and commentators.

\section*{Significance for the Project}
This article provides a valuable methodology for analyzing interpersonal influence within social networks, aligning with Promueva's objectives of understanding social dynamics and polarization. The proposed approach can help identify influential users and measure the strength of their connections, providing insights into social influence mechanisms.

\end{document}
