\documentclass{article}
\usepackage{hyperref}
\usepackage{geometry}

\geometry{a4paper, margin=1in}

\title{A Dynamic Trust Network and Influence Measure Based Consensus Model for Large-Scale Group Decision-Making with Incomplete Intuitionistic Fuzzy Preference Relations}
\author{}
\date{}

\begin{document}

\maketitle

\section*{Paper Information}
\begin{itemize}
    \item \textbf{Paper link}: \url{https://doi.org/10.1080/01605682.2023.2237987}
    \item \textbf{Reviewer}: Juan José Viafara Carabali
    \item \textbf{Categories}: Opinion Evolution, Data
\end{itemize}

\section*{Addressed Problem}
How to calculate influence in social networks with trust relationships and incomplete preferences?

\section*{Main Contribution}
Proposal of a consensus model that measures influence based on two factors: trust between users and opinion similarity.

\section*{Methodology Comments}
The method uses trust and opinion similarity to calculate user influence, making it useful when there is insufficient information about some users.

\section*{General Description}
This paper introduces a model for measuring influence in large social networks with incomplete information. It calculates influence based on trust and similarity of opinions between users, highlighting that influential users are trusted by others.

\section*{Significance for the Project}
This model offers insights for Promueva by showcasing a method that measures influence in networks where trust relationships and incomplete preferences exist, potentially assisting in scenarios with limited data on certain user interactions.

\end{document}
